%++++++++++++++++++++++++++++++++++++++++++++++++++++++++++++++++++++++++++++++
\section{Introducción}
\begin{frame}[allowframebreaks,fragile]
  \frametitle{Introducción}
  \begin{center}
    Un lenguaje diseñado para crear DSLs
  \end{center}
  % Comentar el propósito del proyecto. Cual es el objetivo
  % Cual ha sido el planteamiento, para solucionar el problema
  %


  \begin{center}
    Objetivos generales
    % Se han planteado diversas metas a alcanzar para la realización de este proyecto.

    \begin{itemize}
      \item REPL
        % El repl es una de las principales metas debido a que permite aprendizaje
        % más eficaz que el equivalente mediante scripts. Además de su uso en la
        % depuracion del propio código del intérprete, mediante los comandos especiales.
        %
      \item Scripts
        % Posibilidad de realizar script. Para realizar las tareas de automatización.
        %
      \item Archivos de configuracion
        % Las cuales puedan tener archivos de configuración asociados. Es decir existen mecanismos para
        % generar configuraciones específicas para que se soporten DSLs.
        %
      \item Interfaz con Github
        % La interfaz con github pretende ser un ejemplo del potencial de la sintaxis y las capacidades
        % internas del lenguaje
        %
    \end{itemize}
  \end{center}

  \framebreak

  \frametitle{Estado}
  \begin{center}
    Estado
  \end{center}

  \framebreak

  \begin{center}
    Limitación de los scripts
    \begin{itemize}
      \item Linux, con posible soporte a MacOS
        % Actualmente solo, tiene soporte para linux, y se prevee que en macos funcione sin problemas mayores
        % en cuanto al soporte para windows esta limitado por la dependencia de pcre. La cual se puede parchear
        % para su posible port a windows.
        %
      \item Limitada cantidad de operaciones sobre objetos
        % Los métodos implementados son una cantidad, bastante baja. La implementación de más no debería ser ningún problema.
        %
      \item Manejo de la memoria
        % La memoria está implementada para ser fácil de depurar. No se ha tenido en consideración la eficacia de la misma, hasta establecer
        % un modelo bastante robusto. Y ciertas caracteristicas del interprete se estabilizen
        %
    \end{itemize}
  \end{center}

  \framebreak
  \begin{center}
    REPL
    \begin{itemize}
      \item Básica funcionalidad del REPL
      \item Diversos comandos implementados
      \item Permite insersión de codigo multilínea
    \end{itemize}
  \end{center}

  \framebreak
  \begin{center}
    Esquema interno
    \begin{itemize}
      \item Interoperabilidad
      \item AST flexible
      \item Reuso de funciones de Haskell
    \end{itemize}
  \end{center}
\end{frame}


\section{Usos}
\begin{frame}[allowframebreaks,fragile]
  \frametitle{Usos}
  \begin{center}
    Estado
    \begin{itemize}
      \item Interoperabilidad
      \item AST flexible
      \item Reuso de funciones de Haskell
    \end{itemize}
  \end{center}

  \frametitle{REPL}


  \begin{lstlisting}[language=scriptflow]
    use Github()
    cd virtual_github

    for repo in logued_user.repos & r/TFG-.*/:
      cd repo
      linter_info = !> jslinter .

      if linter_info.status == 2:
        templateIssue = Issue {
          title -> "Linter Fails",
          message -> "Doesn't pass linter:\n" ++ linter_info.output,
          assignees -> [ logued_user ] ++ repo.collaborators,
        }
        repo.new_issue templateIssue
  \end{lstlisting}


\end{frame}

\section{Conclusiones y uso futuro}
\begin{frame}[allowframebreaks,fragile]
  \frametitle{Conclusión}
  \begin{center}
    \begin{itemize}
      \item Implementación de la API de Github
      \item Mejora
    \end{itemize}
  \end{center}

  \frametitle{Conclusions and future work}
  \begin{center}
    \begin{itemize}
      \item  Default item label for entry one
      \item  Default item label for entry two
      \item  Custom item label for entry three
    \end{itemize}
  \end{center}

\end{frame}
