%++++++++++++++++++++++++++++++++++++++++++++++++++++++++++++++++++++++++++++++
\section{Introducción}
\begin{frame}[allowframebreaks,fragile]
  \frametitle{Introducción}
  \begin{center}
    Un lenguaje diseñado para crear DSLs
  \end{center}
  % Comentar el propósito del proyecto. Cual es el objetivo
  % Cual ha sido el planteamiento, para solucionar el problema
  %
\end{frame}
\section{Objetivos}
\begin{frame}[allowframebreaks,fragile]
  \frametitle{Objectivos}
  \begin{center}
    Objetivos generales
    % Se han planteado diversas metas a alcanzar para la realización de este proyecto.

    \begin{itemize}
      \item REPL
        % El repl es una de las principales metas debido a que permite aprendizaje
        % más eficaz que el equivalente mediante scripts. Además de su uso en la
        % depuracion del propio código del intérprete, mediante los comandos especiales.
        %
      \item Scripts
        % Posibilidad de realizar script. Para realizar las tareas de automatización.
        %
      \item Archivos de configuracion
        % Las cuales puedan tener archivos de configuración asociados. Es decir existen mecanismos para
        % generar configuraciones específicas para que se soporten DSLs.
        %
      \item Interfaz con Github
        % La interfaz con github pretende ser un ejemplo del potencial de la sintaxis y las capacidades
        % internas del lenguaje
        %
    \end{itemize}
  \end{center}

  \framebreak{}

\end{frame}
\section{Estado}
\begin{frame}[allowframebreaks,fragile]
  \frametitle{Estado}
  \begin{center}
    Limitación del intérprete
    \begin{itemize}
      \item Linux, con posible soporte a MacOS
        % Actualmente solo, tiene soporte para linux, y se prevee que en macos funcione sin problemas mayores
        % en cuanto al soporte para windows esta limitado por la dependencia de pcre. La cual se puede parchear
        % para su posible port a windows.
        %
      \item Limitada cantidad de operaciones sobre objetos
        % Los métodos implementados son una cantidad, bastante baja. La implementación de más no debería ser ningún problema.
        %
      \item Manejo de la memoria
        % La memoria está implementada para ser fácil de depurar. No se ha tenido en consideración la eficacia de la misma, hasta establecer
        % un modelo bastante robusto. Y ciertas caracteristicas del interprete se estabilizen
        %
    \end{itemize}
  \end{center}

  \framebreak{}
  \begin{center}
    REPL
    \begin{itemize}
      \item Básica funcionalidad del REPL
        % Tiene las funcionalidades básicas esperadas de un REPL
        % tales como historial, un autocompletado simple, keybinds de emacs
      \item Diversos comandos implementados
        % Como mem y instrs principalmente para la propia depuración del lenguaje
        % se tiene pensado añadir más comandos. como tutorial o docs que permitan explotar
        % más estacaracterisca
        % Hay funciones implementadas se podrían considerar comandos como dir. El cual permite obtener
        % lista de los atributos y métodos de un objecto
      \item Permite insersión de codigo multilínea
        % El REPL intenra parsear el codigo al ejecutar el retorno de carro en caso de que se detecte un corchete abierto
        % inicia el modo multilinea hazta que entra una linea vacia. Lo que ejecutaría el codigo mencionado
        \begin{lstlisting}[language=scriptflow]
          >>> if $cat somefile.txt$.exec().null():
          ...   print "It is empty"
          ...
          It is empty
          none
          >>> |
        \end{lstlisting}
    \end{itemize}
  \end{center}

  \framebreak{}
  \begin{center}
    Esquema interno
    \begin{itemize}
      \item Interoperabilidad
        % Se trata de una característica limita entorno de haskell y unidireccional
        % es decir se usa desde el propio codigo de haskell y permite usar codigo haskell dentro
        % de scriptflow. Pero no lo contrario, desde el propio REPL de scriptflow llamar directamente
        % a codigo haskell.

      \item AST flexible
        % El AST recientemente se a cambiado uno más flexible encontrado en el paper
        % Tree that grows. Este usa diversas extensiones del compilador GHC. Para lograr esta
        % estructura, familias de tipo abiertas.
      \item Reuso de funciones de Haskell
        % El proceso reusar las funciones de haskell es un tema con dificultad debido
        % a la propia diferencia entre lenguajes, tipado estrito vs dinamico, funciones currificadas
        %
        \begin{minted}[fontsize=\small]{haskell}
          methodsTh
            [ fn "init" [| T.init :: T.Text -> T.Text |]
            , fn "null" [| T.null :: T.Text -> Bool |]
            ]
        \end{minted}
    \end{itemize}
  \end{center}
\end{frame}


\section{Usos}
\begin{frame}[allowframebreaks,fragile]
  \frametitle{Usos}
  \begin{center}
    \begin{itemize}
      \item Pequeños scripts.
      \item Scripts con configuraciones
      \item REPL
    \end{itemize}
  \end{center}
  \framebreak{}
  \begin{center}
    %% TODO Realizar ejemplos que permitan simplificar cierto trabajo
    %%      de la shell
    %% TODO Ejemplos de interación con la shell
    \begin{lstlisting}[language=scriptflow]
      class Utils:
        fun ...
    \end{lstlisting}
  \end{center}

\end{frame}

\section{Conclusiones y futuras mejoras}
\begin{frame}[allowframebreaks,fragile]
  \frametitle{Conclusión y futuras mejoras}
  \begin{center}
    \begin{itemize}
      \item Implementación de la API de Github
      \item Mejorar la implementación de la memoria
      \item Poner en práctica el posible uso de FUSE en el DSL de Github
      \item Muchas posibles mejoras y caracteristicas
    \end{itemize}
    % Elementos a resaltar de una proxima versión
    % - nueva sintaxis de regex
    % - nueva sintaxis para comandos de línea
    % - metodo especial __cd__, __filter__
    % - pipes
    \begin{lstlisting}[language=scriptflow2]
      use Github()

      for repo in logued_user.repos | r/tfg-.*/:
        cd repo; use repo
        linter_info = !$ jslinter .

        if linter_info.status == 2:
          templateissue = Issue {
            title -> "linter fails",
            message ->
              "Doesn't pass linter:\n" ++ linter_info.output,
            assignees ->
              [ logued_user ] ++ collaborators,
          }
          new_issue templateissue
    \end{lstlisting}
  \end{center}
\end{frame}

\begin{frame}[allowframebreaks,fragile]
  \frametitle{Conclusions and future work}
  \begin{center}
    \begin{itemize}
      \item Implement Github API as DSL
      \item Improve current implemented memory manager
      \item Make prototype using libfuse in Github DSL
      \item A lot more of improvements and features
    \end{itemize}
    \begin{lstlisting}[language=scriptflow2]
      use Github()

      for repo in logued_user.repos | r/tfg-.*/:
        cd repo; use repo
        linter_info = !$ jslinter .

        if linter_info.status == 2:
          templateissue = Issue {
            title -> "linter fails",
            message ->
              "Doesn't pass linter:\n" ++ linter_info.output,
            assignees ->
              [ logued_user ] ++ collaborators,
          }
          new_issue templateissue
    \end{lstlisting}
  \end{center}

\end{frame}
