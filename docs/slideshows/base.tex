\documentclass{beamer}
%\documentclass[xcolor=dvipsnames]{beamer}
\usepackage[spanish]{babel}
\usepackage[utf8]{inputenc}
\usepackage{graphicx}
\usepackage{latexsym}
\usepackage{listings}
\usepackage{color}
\usepackage{amssymb}

\newcommand{\beamer}{\textsc{beamer}}

%\newtheorem{definicion}{Definición}
%\newtheorem{ejemplo}{Ejemplo}

%%%%%%%%%%%%%%%%%%%%%%%%%%%%%%%%%%%%%%%%%%%%%%%%%%%%%%%%%%%%%%%%%%%%%%%%%%%%%%%
\title[Trabajo de Fin de Grado]{
  Un lenguaje para la creación de DSLs. \\
  A language to making DSLs.
}

\author[Eleazar Díaz Delgado] {
Autor: Eleazar Díaz Delgado \\
Director: Casiano Rodríguez León
}

\institute[ULL]{Escuela Superior de Ingeniería y Tecnología \\
                Departamento de Ingeniería Informática y de Sistemas \\
                Universidad de La Laguna}
\date[07-09-2018]{7 de Septiembre de 2018}
%%%%%%%%%%%%%%%%%%%%%%%%%%%%%%%%%%%%%%%%%%%%%%%%%%%%%%%%%%%%%%%%%%%%%%%%%%%%%%%

%\usetheme{Berlin}
\usetheme{Madrid}

%%%%%%%%%%%%%%%%%%%%%%%%%%%%%%%%%%%%%%%%%%%%%%%%%%%%%%%%%%%%%%%%%%%%%%%%%%%%%%%
\definecolor{pantone254}{RGB}{92,6,140}
\definecolor{pantone3015}{RGB}{92,6,140}
\definecolor{pantone432}{RGB}{92,6,140}
\setbeamercolor*{palette primary}{use=structure,fg=white,bg=pantone254}
\setbeamercolor*{palette secondary}{use=structure,fg=white,bg=pantone3015}
\setbeamercolor*{palette tertiary}{use=structure,fg=white,bg=pantone432}
\setbeamercolor*{palette sidebar primary}{use=structure,fg=pantone254}
\setbeamercolor*{palette sidebar tertiary}{use=structure,fg=pantone3015}
\setbeamercolor*{block title}{bg=pantone3015,fg=white}
\setbeamercolor*{alerted text}{fg=pantone432}
\setbeamercolor*{item}{fg=pantone254} % projected
\setbeamercolor*{section in toc shaded}{use=structure,fg=structure.fg}
\setbeamercolor*{section in toc}{fg=pantone3015}
\setbeamercolor*{subsection in toc shaded}{fg=pantone3015}
\setbeamercolor*{subsection in toc}{fg=pantone432}

% I don't like perfect rounded latex circles
\setbeamertemplate{itemize items}{$\blacksquare$}
\setbeamertemplate{section in toc}[square]


%%%%%%%%%%%%%%%%%%%%%%%%%%%%%%%%%%%%%%%%%%%%%%%%%%%%%%%%%%%%%%%%%%%%%%%%%%%%%%%


\lstdefinelanguage{scriptflow}{
  sensitive = true,
  keywords={class, fun, lam, for, in, if, else},
  otherkeywords={% Operators
    >, <, ==, <=, >=, ->, **, *, /, \%, +, -, \&\&, ||, !, @, :
  },
  keywords = [2]{true, false, none},
  keywordstyle=\color[HTML]{8700af},
  keywordstyle=[2]\color[HTML]{005faf}, % for example
  showstringspaces=false,
  breaklines=true,
  comment=[l]{\#},
  commentstyle=\color[HTML]{878787}\ttfamily,
  stringstyle=\color[HTML]{5f8700}\ttfamily,
  morestring=[b]",
  morestring=[b]$
}

%%%%%%%%%%%%%%%%%%%%%%%%%%%%%%%%%%%%%%%%%%%%%%%%%%%%%%%%%%%%%%%%%%%%%%%%%%%%%%%
\begin{document}
% Cover
\begin{frame}

  \includegraphics[width=0.3\textwidth]{img/ull.eps}
  \hspace*{7.5cm}
  \titlepage

\end{frame}

% Index
\begin{frame}
  \frametitle{Índice}
  \tableofcontents
\end{frame}

% Slides
%++++++++++++++++++++++++++++++++++++++++++++++++++++++++++++++++++++++++++++++
\section{Introducción}
\begin{frame}[allowframebreaks,fragile]
  \frametitle{Introducción}
  % Comentar el propósito del proyecto. Cual es el objetivo
  % Cual ha sido el planteamiento, para solucionar el problema
  %

  \begin{center}
    Propósito
  \end{center}

  \framebreak

  \begin{center}
    Estado
  \end{center}

  \begin{center}
    Capacidades
  \end{center}

\end{frame}


\section{Usos}
\begin{frame}[allowframebreaks,fragile]
  \frametitle{}

  \begin{lstlisting}[language=scriptflow]
  # A comment
  class Test:
    fun test arg1 arg2:
      test =
        if val > fd:
          r"hola"
          "fdfasdf"
          !$echo$
          89
        else:
          { hola -> 45.5 }
  \end{lstlisting}


\end{frame}

\section{Conclusiones y uso futuro}
\begin{frame}[allowframebreaks,fragile]
  \frametitle{Conclusión}
  \begin{center}
    \renewcommand{\labelitemi}{$\blacksquare$}
    \renewcommand\labelitemii{$\square$}
    \begin{itemize}
      \item  Default item label for entry one
      \item  Default item label for entry two
      \item  Custom item label for entry three
    \end{itemize}
  \end{center}

  \frametitle{Conclusions and future use}

\end{frame}




% Bibliography
\section{Bibliografía}
\begin{frame}[allowframebreaks]
  \frametitle{Bibliografía}
  \bibliographystyle{unsrt}
  \bibliography{../manuscript}
  \nocite{*}
\end{frame}

\begin{frame}
  \frametitle{Fin de la presentación}
  \begin{center}
    \Huge{Gracias por su atención}
  \end{center}
\end{frame}
\end{document}
