% Created 2018-09-03 lun 23:03
% Intended LaTeX compiler: pdflatex
\documentclass[11pt]{article}
\usepackage[utf8]{inputenc}
\usepackage[T1]{fontenc}
\usepackage{graphicx}
\usepackage{grffile}
\usepackage{longtable}
\usepackage{wrapfig}
\usepackage{rotating}
\usepackage[normalem]{ulem}
\usepackage{amsmath}
\usepackage{textcomp}
\usepackage{amssymb}
\usepackage{capt-of}
\usepackage{hyperref}
\usepackage[version=3]{mhchem}
\usepackage{float}
\author{Eleazar Díaz Delgado }
\date{\today}
\title{}
\hypersetup{
 pdfauthor={Eleazar Díaz Delgado },
 pdftitle={},
 pdfkeywords={},
 pdfsubject={},
 pdfcreator={Emacs 26.1 (Org mode 9.1.14)},
 pdflang={English}}
\begin{document}

\title{Un lenguaje para realizar DSLs}
\author{Eleazar Díaz Delgado}

\date{La Laguna, 4 de Septiembre de 2018}

\makeatletter
\begin{titlepage}
    \includegraphics[width=40mm]{ull-logo.jpg}\\[4ex]
    \begin{center}
        {\huge \bfseries  Trabajo de Fin de Grado }\\[2ex]
        {\LARGE  Grado en Ingeniería Informática}\\[10ex]
    \end{center}
    \begin{flushright}
        {\huge \bfseries  \@title }\\[2ex]
        {\huge  A language to make DSLs }\\[2ex]
        {\LARGE  \@author}
    \end{flushright}
    \vfill
    \begin{center}
        {\LARGE \@date}
    \end{center}

\end{titlepage}
\makeatother
\thispagestyle{empty}
\newpage


D. Casiano Rodríguez León, con N.I.F. 42.020.072-S profesor Titular de Universidad adscrito al Departamento de Ingeniería Informática y de Sistemas de la Universidad de La Laguna, como tutor

{\large \bfseries C E R T I F I C A (N)}

Que la presente memoria titulada: “Un lenguaje para realizar DSLs"

ha sido realizada bajo su dirección por D. Eleazar Díaz Delgado
con N.I.F. 54.117.199-Q.

Y para que así conste, en cumplimiento de la legislación vigente y a los efectos oportunos firman la presente en La Laguna, 4 de Septiembre de 2018.
\newpage
\begin{flushright}
    {\huge  Agradecimientos }\\[2ex]
    A mi familia por el apoyo, \\
    y en especial a mi madre por preguntarme \\
    casi todos los días por el estado del TFG. \\
    \vspace{10mm} %5mm vertical space
    También dar la gracias a los profesores \\
    a lo largo de este grado y en especial a Casiano. \\
\end{flushright}


\newpage
\begin{flushleft}
  {\huge  Licencia }\\[2ex]
\end{flushleft}

\begin{center}
  \includegraphics[width=40mm]{license.png}\\[4ex]
\end{center}
© Esta obra está bajo una licencia de Creative Commons Reconocimiento-NoComercial 4.0 Internacional

\newpage
\begin{center}
  {\huge  Resumen }\\[2ex]
\end{center}

{\fontsize{14}{11}\selectfont
   ScriptFlow es un lenguaje de tipado dinámico para el desarrollo de scripts para la automatización de tareas, que requieran configuraciones.
   Se trata de un lenguaje basado en expresiones que da la opción a ser sensible a la indentación. Incluye una integración con Haskell por el
   cual puede ser ampliable.
}\\
\vspace{50mm}
\textbf{Palabras clave}: ScriptFlow, Intérprete, Compilador, Haskell, DSL.

\newpage
\begin{center}
  {\huge  Abstract }\\[2ex]
\end{center}

{\fontsize{14}{11}\selectfont
   ScriptFlow is a dynamic typed language to develop scripts to automatize a sets of task, whose of these requires use of configuration files.
   It is a language based in expressions that allows to you to use identation-sensitive syntax. It is includes a integration with Haskell language,
   which, it was built in.
}\\
\vspace{50mm}
\textbf{Keywords}: ScriptFlow, Interpreter, Compiler, Haskell, DSL.

\newpage

\renewcommand{\contentsname}{Índice general}

\tableofcontents

\newpage


\section{Introducción}
\label{sec:org5c9bb11}

ScriptFlow es un lenguaje de programación interpretado pensado para la creación de lenguajes de dominios específicos. El objetivo del
lenguaje es combinar ficheros de configuración, con características del lenguaje y las APIs. Con el fin de realizar los lenguajes de dominio específico,
que permitan resolver tareas y posteriormente automatizarlas en la medida de lo posible. Estos DSL capaces de realizar el lenguaje creado, estan planteado
para trabajar mediante via textual o terminal. Es decir, no está pensado para realizar una interfaz gráfica que permita automatizar problemas sino mediante
scripts y un REPL customizable según el DSL usado.

\section{Metodología}
\label{sec:org74980ec}

Se usado el lenguaje de programación Haskell, debido a su capacidad para trabajar con ADT, y estructuras abstractas. Permite gestionar
el código mejor que la representación equivalente usando objetos, en un lenguaje POO. Junto con la infraestructura \emph{Stack-Haskell} para
gestionar las dependencias debido a su estabilidad.

La parte de análisis sintáctico y léxico. Se optó por usar dos fases para la generación del AST
debido a la simplificación del problema de la indentación. Se tokeniza con la librería \emph{Alex} y se parsea con \emph{parsec}.
Añadido a esto se usan librerías para gestionar el REPL (Haskeline) o internamente estructuras de datos como pueden ser (mtl, transformers, free, prettyprinters, \ldots{})
entre otras.

En las configuraciones se usa el formato Yaml por permitir crear configuraciones más complejas que el equivalente en JSON.

Para realizar el proyecto se usan las herramientas VsCode y Neovim para el desarrollo del software usando diversos
plugin de entre ellos LSP (Language Server Protocol).

Los recursos son obtenidos de la propia documentación de las librerías, como información adicional que se encuentra en Internet.

\section{Lenguaje ScriptFlow}
\label{sec:org21d94ce}

\subsection{Sintáxis}
\label{sec:orgf5783b0}

ScriptFlow se ha diseñado para ser sencillo de usar, para ello se optado por una sintaxis familiar a lenguajes como
\emph{python} o \emph{c++}.

Los comentarios se comienzan con \texttt{\#} y el retorno de línea delimita su fin. Estos pueden ser añadidos al final de una expresión y no están sujetos a las reglas de indentación
lo que permite cierta flexibilidad.

\begin{verbatim}
# Variable ejemplo
   # Comentario
example = 1 + 1 # Commentario

if example == 2:
# Otro ejemplo
  print "something"
\end{verbatim}

\subsubsection{Identación}
\label{sec:org661b8d3}

\label{org6e76675}
El lenguaje puede usar su sintáxis sensible a la indentación como \emph{python} o \emph{haskell} pero se trata de algo opcional que puede ser omitido usando las llaves "\{\}":

\begin{verbatim}
fun saludar_a nombre:
  print("Saludos, " ++ nombre)
\end{verbatim}

O la sintáxis equivalente al estilo de la familia de \emph{C}:
\begin{verbatim}
fun saludar_a nombre {
  print("Saludos, " ++ nombre)
}
\end{verbatim}

\subsubsection{Literales}
\label{sec:org9a7d8de}

El lenguaje cuenta con diversas primitivas integradas por defecto, como
booleanos, enteros, decimales, cadenas, expresiones regulares, comandos
shell.

\begin{verbatim}

# none
none

# Booleanos
true
false

# Enteros
12 + 4 # ...

# Decimales (Se utiliza un double para su
# representanción actualmente)
45.5

# Las cadenas pueden ser multilineas se crean con
# comillas dobles "
"Un\nejemplo
de cadena"

# Las regex se construyen
# con r" y terminan con ".
# Usan la sintaxis de PCRE.
r"a*"

# Los comandos shell se crean con
$cd #HOME/repos; nvim .$
Ó mutltilinea
$
cd #HOME/repos
nvim .
$

O la sintaxis alternativa
$"pwd"
\end{verbatim}

A su vez también existen tipos contenedores tales como los vectores y los
diccionarios o tablas hash.

\begin{verbatim}
# Vectores
# Los vectores pueden contener diferentes tipos en el mismo vector
[45, "tipos", []]

# Diccionarios
{ test -> [1,2,3,47,5]
  , author ->
  { name -> "Flynn"
  , "vive en" -> "tal sitio"
  }
}

\end{verbatim}

\subsubsection{Expressiones}
\label{sec:org7ee14d5}
El lenguaje esta compuesto por expresiones, es decir, todas las estructuras devuelven algún valor. Estas expresiones, se encuentran delimitadas de forma diferente según en que contexto
se encuentren.
Las expresiones en la base del archivo, tales como;
\begin{verbatim}
print "Hello World"

var = 67

func_call
   first_param
   second_param
\end{verbatim}

Son delimitadas por el final de linea, o en el caso de exista cierto nivel de indentación mayor que el base '0' se agrupan con la primera sin indentación. Es decir, en el caso de \texttt{func\_call}
la expresión final sería \texttt{func\_call(first\_param, second\_param)}. Se puede usar el carácter ';' para realizar esta separación (el cual es opcional al nivel base).

En el caso de expresiones más complejas que requieran un subconjunto de expresiones, hablamos de \texttt{if}, \texttt{for} \ldots{}. Se contemplan dos casos para realizar la terminación de las expresiones.
Si se usa sintáxis sensible a la indentación, los niveles de indentación determinarán donde se halla la terminación de las expressiones. Pero, por si el contrario se usa sintáxis con llaves
se necesitará añadir ';' para indicar la terminación de cada expressión. Y opcionalmente se puede quitar el ';' de la última expressión.

\begin{verbatim}
if always_true:
  make_test test1 test2
  other_func
     arg1
     arg2
  end_test arg_end

if always_true {
  make_test test1 test2;
  other_func
     arg1
     arg2;
  end_test arg_end
}
\end{verbatim}

\subsubsection{Funciones}
\label{sec:org63f114c}

La sintáxis permite definir dos tipos de funciones, aquellas que tienen un nombre y las lambda. Internamente solo hay lambdas debido
a que la primeras son traducidas a una función lambda asignada a una variable.

La sintáxis de las funciones lambda es la siguiente:
\begin{verbatim}
# Con identación
lam arg1 arg2:
   arg1

# O alternativamente
lam arg1 arg2 { arg1 }
\end{verbatim}

Las funciones con nombre, en el siguiente ejemplo;
\begin{verbatim}
fun func_name arg1 arg2 { arg1 }

fun func_name arg1 arg2:
   arg1
\end{verbatim}

\subsection{Orientado a objetos}
\label{sec:org3213ca5}

\subsubsection{Objectos}
\label{sec:orge75dfae}
Un objeto en ScriptFlow es un diccionario con la clase a la que pertenece, en el caso de ser
un objeto instanciado.

En el siguiente ejemplo se enumeran las distintas formas de crear un objeto:

\begin{verbatim}
# A partir de un diccionario vacío
obj = {}

# A partir de none
obj2 = none
# Al asignar dentro de una varible establecidad `none` un "sub-item".
# Automáticamente se genera un objeto con ese ítem dentro
obj2.a = "ejemplo"
> { a -> "ejemplo" }

# A partir de una clase definida
class Test {}
# El constructor devolverá la instancia correspondiente
obj3 = Test()
\end{verbatim}

Los objetos tienen diversas características incorporadas con el intérprete para mejorar su uso dentro de una DSL.

Las funciones \texttt{use} y \texttt{unuse} permiten modificar el ámbito actual de búsqueda de variables, y simplificar ciertos escenarios.

La función \texttt{use} genera un nuevo ámbito que queda detrás del actual permitiendo acceder a los attributos y funciones directamente
sin necesidad de especificar a que objecto se refiere. Las nuevas variables creadas dentro del ámbito sobreescriben las creadas por \texttt{use}
debido a que continuán en el ámbito superior. La resolución de nombres al usar \texttt{use} sobre un objeto, tiene la menor precedencia
dentro de la propia resolución del nombres, y la última llamada de \texttt{use} tiene mayor precedencia que las anteriores de use.

La función \texttt{unuse} deshace el último \texttt{use} usado. Se tiene planeado en futuras versiones realizar automaticamente un \texttt{unuse} al salir de un ámbito.

Un ejemplo ilustrativo de como trabaja esta funcionalidad dentro de un DSL.

\begin{verbatim}
class Github:
  fun repositories {} # return a list of repositories
  fun user_name {}
class Repository:
  fun name {}
  fun issues {}

gh = use Github()
filter_reg = Regex gh.user_name
for repo in repositories:
  use repo
  print name
  print issues.filter(filter_reg)
  unuse
unuse
\end{verbatim}

\subsubsection{Clases}
\label{sec:org71ea073}
El lenguaje tiene un básico soporte a la programación orientada a objetos. Permite la definición
de clases sin la capacidad de herencia. El siguiente ejemplo sobrecarga el constructor de la clase
usando el método especial \texttt{\_\_init\_\_}.

Los métodos asociados al objeto internamente se pasan a si mismo como argumento usando la palabra
reservada \texttt{self}. El lenguaje no soporta métodos estáticos.


\begin{verbatim}
class Repository {
        fun __init__ new_name {
            self.url = none
            self.local_repo = none
            self.name = new_name
        }
}
\end{verbatim}

La siguiente tabla muestra los métodos disponibles para sobrecargar.

\begin{center}
\begin{tabular}{lrll}
Operador & Nivel de precedencia & Precedencia & Nombre método\\
\hline
** & 8 & Izquierda & \texttt{\_\_pow\_\_}\\
\texttt{*} & 7 & Izquierda & \texttt{\_\_mul\_\_}\\
\texttt{/} & 7 & Izquierda & \texttt{\_\_div\_\_}\\
\% & 7 & Izquierda & \texttt{\_\_mod\_\_}\\
+ & 6 & Izquierda & \texttt{\_\_plus\_\_}\\
- & 6 & Izquierda & \texttt{\_\_minus\_\_}\\
++ & 5 & Derecha & \texttt{\_\_append\_\_}\\
\texttt{==} & 4 & Izquierda & \texttt{\_\_eq\_\_}\\
!= & 4 & Izquierda & \texttt{\_\_neq\_\_}\\
/= & 4 & Izquierda & \texttt{\_\_neq\_\_}\\
> & 4 & Izquierda & \texttt{\_\_gt\_\_}\\
< & 4 & Izquierda & \texttt{\_\_lt\_\_}\\
<= & 4 & Izquierda & \texttt{\_\_le\_\_}\\
>= & 4 & Izquierda & \texttt{\_\_ge\_\_}\\
\&\& & 3 & Derecha & \texttt{\_\_and\_\_}\\
\(\vert{} \vert{}\) & 3 & Derecha & \texttt{\_\_or\_\_}\\
\texttt{!} & 1 & Izquierda & \texttt{\_\_not\_\_}\\
@ & 1 & Izquierda & \texttt{\_\_at\_\_}\\
print & - & -- & \texttt{\_\_print\_\_}\\
\end{tabular}
\end{center}


El método especial \texttt{\_\_print\_\_} indica la forma visualización, que debe mostrarse por pantalla el objecto al usar la función \texttt{print}.


\section{Configuración}
\label{sec:orgb5495bd}

\label{org38c8232}
El fichero de configuración se localiza mediante el
estándar XDG. Normalmente localizado en \texttt{/home/username/.config/scriptflow}
La configuración es un fichero tipo YAML. El cual permite especificar
parámetros de configuración, tales como el prompt, shell. O parametros
específicos con la API Web; tales como la autenticación o posibles
preferencias.

\subsection{Prompt}
\label{sec:orge9698d7}

\label{org1230f1c}
En el modo interactivo del intérprete (repl) permite la personalización del
\textbf{prompt}. Tales como la salida de la ejecución de comandos
shell, y diversos comandos propios del intérprete. La configuración del
prompt se puede realizar desde el fichero de configuración (véase:
\ref{org38c8232}) en la sección \textbf{repl}.

Por defecto, la sección del \emph{prompt} contiene la siguiente configuración:

\begin{verbatim}
repl:
  # ...
  prompt: |
     $"pwd".exec().strip() ++ " >>> "
  # ...
\end{verbatim}

La configuración del prompt debe ser una expresión de ScriptFlow.

\section{REPL}
\label{sec:orgae75343}
El \textbf{REPL} puede ser accedido mediante comando de líneas \texttt{scriptflow} o con la
finalización de ejecución de un \textbf{script} con la opción \texttt{-e}. Se pueden ver más opciones del
ejecutable del intérprete mediante \texttt{scriptflow -{}-help}. Una vez,
iniciado el \textbf{REPL} se mostrará por defecto el \textbf{prompt} predeterminado
(configuración véase: \ref{org1230f1c}).

Desde el \textbf{REPL} se puede escribir cualquier tipo de expresión definida por el
lenguaje. Y los comandos del intérprete los cuales comienzan por ":". Se
puede ver una lista de los comandos con \texttt{:help}

\begin{itemize}
\item \texttt{:instr}

Permite visualizar, a que instrucciones se traduce el código. Estas
instrucciones son parciales solo sirven de guía. (Véase: \ref{orgd492490})

\item \texttt{:mem}

Muestra parcialmente las variables disponibles en memoria.

\item \texttt{:quit}

Sale del intérprete.
\end{itemize}

\section{Arquitectura del proyecto}
\label{sec:org5aa4036}

\subsection{Introducción}
\label{sec:org9d4bce0}

El lenguaje se ha realizado usando un lenguaje puramente funcional lo que
requiere diferentes enfoques al realizar el diseño del interprete. Ya que
no posée una interfaz orientada a objetos. Dada esta diferencia voy a
detallar en cierta medida peculiaridades del desarrollo, en las siguientes
secciones. Antes de ello empezaremos con un pequeño análisis de como
funciona el intérprete.

Dado un fichero de entrada con el código escrito en ScriptFlow.

\begin{verbatim}
fun say_hi name:
  "Hola, " ++ name

say_hi("Mundo")
\end{verbatim}


Se procede al \emph{parseo} del código, el cual, se realiza a dos fases. La primera el
\emph{tokenizador}, se encarga de transformar, el texto en de entrada, en una
secuencia de \emph{tokens}. Estos tokens representan los elementos importantes
que se usarán para generar el AST (Abstract Syntax Tree). Cada \emph{token} contiene la información
necesaria para reconstruir la parte esencial del código.

\begin{verbatim}
[FunT, NameIdT "say_hi", NameIdT "name", OBraceT,
      LitTextT "Hola, ", OperatorT "++", NameIdT "name",
CBraceT,
NameIdT "say_hi",OParenT,LitTextT "Mundo",CParenT]
\end{verbatim}

En esta fase de \emph{tokenización}, se procede a identificar los niveles de
indentación en el código en el caso necesario (Para más información ir: \ref{org6e76675}).
El \emph{tokenizador} procede a añadir las llaves necesarias en el caso de usar la
gramática del lenguage sensible al contexto. Estos \emph{tokens} se identifican con
\texttt{OBraceT} y \texttt{CBraceT}.

La segunda fase del \emph{parseo} se encarga de generar el árbol sintáctico
abstracto (AST).

\begin{verbatim}
SeqExpr [
      VarDecl (Simple "say_hi")
                (FunDecl ["name"]
                    (SeqExpr
                      [Apply (Simple "++")
                          [Factor (AStr "Hola, "),
                          Identifier (Simple "name")]
                      ]
                    )
                  )
      ,
      Apply (Simple "say_hi")
            [SeqExpr [Factor (AStr "Mundo")]]
]
\end{verbatim}

La salida del AST está simplificada en este ejemplo, se puede ver una salida más detallada,
añadiendo una mayor verbosidad \texttt{scriptflow -v} (Ver \texttt{scriptflow -{}-help} para más información).

Este proceso se realiza mediante un \emph{parser combinador}, el cual se comporta de
forma parecida a los PEGs. Un ejemplo simplificado es la definción de
una función:

\begin{verbatim}
parseFunDecl :: TokenParser Expression
parseFunDecl = do
  funT
  funName <- nameIdT
  params  <- many nameIdT
  prog    <- parseBody
  return (FunDecl funcName params prog)
\end{verbatim}

Una vez generado se realiza la comprobación del \textbf{scope} del AST. En esta fase
comprueban si están usando variables que no existen, o si sobreescriben
otra. Y se procede al renombrado de las variables.

\begin{verbatim}
SeqExpr [
      VarDecl var_0        -- say_hi
                 (FunDecl [param_0] -- name
                    (SeqExpr
                       [Apply op_0   -- "++"
                          [Factor (AStr "Hola, "),
                           Identifier param_0]
                       ]
                    )
                  )
      ,
      Apply var_0     -- say_hi
            [SeqExpr [Factor (AStr "Mundo")]]
]
\end{verbatim}

Una de la últimas fases es la conversion del AST al conjunto de
instrucciones simplificado. (Vease: \ref{orgd492490})

\begin{verbatim}
Assign var_0
    OFunc [param_0]
          CallCommand op_0 ["Hola, ", GetVal param_0]

CallCommand var_0 ["Mundo"]
\end{verbatim}

Y de esta foma es como se representa el código en memoria. Es decir, las
funciones que se definan su contenido es guardado en este formato.

\subsection{Árbol abstracto sintáctico}
\label{sec:org7d4741d}

El AST (Abstract Syntax Tree) de ScriptFlow ha pasado por diversos cambios en el transcurso del proyecto. Inicialmente
se considero usar el modelo conceptual que se aplica en el paquete "language-haskell-ext" el cual codifica el AST de forma genérica
para que en cada nodo se encuentre el componente genérico. Este componente, se fija en el AST a lo largo de todos los nodos lo que
que conlleva a crear un componente complejo e innecesario en la mayoría de los nodos. Se crea un AST poco flexible.

La solución a este problema se encontró dentro de los \emph{papers} que estan siendo implementados en el propio \emph{GHC}. En el \emph{paper} \cite{shayan-2017-trees}
se decribe como se logra una estructura de datos maś flexible que la convencional. Que por medio de los tipos de familia abiertos (Open Family types)
se logra modificar individualmente el tipo de dato complementario en cada nodo del AST según que fase del compilador se encuentre.

\subsection{Lenguaje intermedio}
\label{sec:orgaf90169}

\label{orgd492490} La última fase es la conversión del AST (Abstract Syntax Tree) en conjunto de instrucciones
que se usarán, para describir las secuencia de acciones. Para llevar acabo la ejecución de un script de ScriptFlow
Este conjunto de instrucciones se encuentra expresado en un ADT (Abstract Data Tree), de tal forma que encaje con la estructura
de datos mónada libre (\emph{Free Monad}) \cite{free-monads}. Este estructura, secuencia las instrucciones y permiten usar la notación \emph{do} de Haskell.

\subsection{Interoperabilidad}
\label{sec:orgc2eba29}

La metaprogramación ha supuesto una simplificación en la comunicación entre lenguaje padre e hijo. Con el fin de reutilizar las funciones
ya testeadas de Haskell, en ScriptFlow. Únicamente realizando cambios oportunos, como el orden de los argumentos.

El desarrollo de esta característica se basa en la definición de un isomorfismo entre los tipos de datos de haskell y los de ScriptFlow.
este isomorfismo se encuentra en las clases de tipo \texttt{FromObject} y \texttt{ToObject}.

Apesar de este isomorfismo, existe una dificultad añadida debido a que las funciones en Haskell son currificadas. Por ejemplo dada la siguiente
función \texttt{f} que recibe dos parametros y retorna un \texttt{Bool}.

\begin{verbatim}
f :: Int -> Int -> Bool
\end{verbatim}

Se debe eliminar esta currificación, para que el tipo concuerde con algo más uniforme.

\begin{verbatim}
f :: [Int] -> Bool
\end{verbatim}

La primera solución, que resuelve el problema, se hizo mediante clases de tipos.
\begin{verbatim}
class Normalize a
   normalize :: a -> [Object] -> Object

instance ToObject a => Normalize a where
   normalize = -- implementación omitida

instance (ToObject a, Normalize r) => Normalize (a -> r) where
   normalize = -- implementación omitida
\end{verbatim}

Las cuales mediante el uso de la recursividad entre instancias de las clases de tipos se resolvía el problema. Sin embargo
el método no es eficiente. Y requiere de una clase auxiliar para contar el número de argumentos que posée una función, la cual use
sobrelapamiento entre instancias \cite{overlaping-instances}.

La opción actual reside crear los \emph{wrappers} a medida para cada función convertida. Para ello se implementado una solución
basada en el uso de la meta-programación conocida en Haskell por \emph{Template Haskell} \cite{template-haskell} .

Ejemplo de código auto-generado, dada la función:
\begin{verbatim}
(>) :: Int -> Int -> Bool
(>) = -- implementación omitida
\end{verbatim}

La salida obtenida es:
\begin{verbatim}
greaterThan :: [Object] -> StWorld Object
greaterThan objs =
  let expectedArgs = 2
      givenArgs    = length objs
  case compare givenArgs expectedArgs of
    LT -> throw $ NumArgsMissmatch expectedArgs givenArgs
    GT -> throw $ NumArgsMissmatch expectedArgs givenArgs
    EQ -> do
      let [arg1, arg2] = objs
      val1 <- fromObject arg1
      val2 <- fromObject arg2
      toObject ((>) val1 val2)
\end{verbatim}

Una de las desventajas de esta solución se encuentra en las propias limitaciones del \emph{Template Haskell}. Debido a que no es posible
inferir el tipo de una expresión dada, lo que requiere añadir el tipo de la expressión.
\begin{verbatim}
$(normalize [| (>) :: Int -> Int -> Bool |])
            -- Se repite el tipo obligatoriamente
\end{verbatim}

La implementación de la "meta-función" se encuentra en el módulo \emph{Compiler.Prelude.Th}.

Otro factor de interoperabilidad a destacar, es la creación de un \emph{QuasiQuoter} \cite{quasi-quoter}. Lo que permite incrustar fragmentos de ScriptFlow
dentro de Haskell. Y dentro del propio \emph{QuasiQuoter} realizar llamadas a funciones de Haskell usando el mecanismo anteriormente descrito
para la conversión de funciones entre ambos lenguajes.

\begin{verbatim}
requestLogin :: String -> String -> IO ()
requestLogin = -- se omite implementación

githubClassSC :: Interpreter Object
githubClassSC = [scriptflow|
    # Github base class
    class Github:
        fun login:
          print "Logging to get authorization token to use in future connections"
          username = get_line "User Name: "
          password = ask_password "*" "Password: "
          __call__ ${requestLogin} username password
  |]
\end{verbatim}

\section{Conclusiones}
\label{sec:org44b7255}

ScriptFlow trata de simplificar el proceso de crear script de automatización, proveeyendo una interfaz unificada entre ficheros
configuración, funciones de interacción con APIs y característcas propias del lenguaje.

Existen diversos problemas y dificultades en el desarrollo de software dentro de la plataforma de Haskell, debido a ser un lenguaje
con una comunidad menor a lenguajes más populares. Se encuentran escasas herramientas de programación o poco actualizadas a las últimas versiones de la plataforma.
Cabe a destacar que la mayor dificultad encontrada es trabajar con dependencias cíclicas entre módulos en Haskell. Es cierto, que existen soluciones pero
no son prácticas para un desarrollo ágil.

Existen diversas mejoras a aplicar sobre el proyecto:
\begin{itemize}
\item Actualmente es solo extensible via Haskell (lo que requiere tener el compilador), una mejora sería permitir interoperabilidad con Python.
\item Los objectos básicos, contienen pocos métodos con los que interactuar entre sí.
\item Para el DSL de Github, se tiene planteado realizar un uso de HFuse. Para simular virtualmente los repositorios de Github en el sistema de ficheros
y con determinadas acciones clonar directamente repositorio, por ejemplo.
\end{itemize}

\section{Conclusions}
\label{sec:org4cc9efc}

ScriptFlow try to simplifies the process of make scripts to automatize, providing with a unified interface between; configuration files, functions that interact with external APIs and
characteristics of the own language.

There are several problems and difficulties developing software into the Haskell platform, due to be a language with smaller community than other popular languages. There are less programming
tools or they are no update to latest versions of platform. I highlight, the greater problem found into this project was cyclic-dependencies between modules in Haskell. It can be solved but there are not optimal
solutions to an agile development.

Exist different improvements to apply over this project:
\begin{itemize}
\item Currently, it can be only extended using Haskell (It requires have a Haskell compiler), it could improve if it allows to inter-operate with Python also.
\item The basic objects, contains few methods to be used between themselves.
\item In the case of Github's DSL, its planed to use HFuse library. For example; to simulate Github repositories virtuality into file system and with specific actions clone this repositories.
\end{itemize}

\section{Presupuestos}
\label{sec:orgd8b36ee}

Se ha realizado una estimación de los costes realizados en el proyecto.

\begin{center}
\begin{tabular}{ll}
Tipo & Descripción\\
\hline
 & \\
\end{tabular}
\end{center}

\renewcommand{\refname}{Bibliografía}

\newpage

\bibliographystyle{unsrt}
\bibliography{manuscript}
\end{document}
